\documentclass[11pt,towside]{article}
\usepackage[brazilian]{babel}			% Ajusta a língua para Portugês Brasileiro
\usepackage{amsmath}					% Permite escrever fórmulas matemáticas
\usepackage[utf8]{inputenc}			% Lê o arquivo fonte codificado em UTF-8
\usepackage[T1]{fontenc}				% Codifica os tipos no arquivo de saída usando T1
\usepackage[a4paper]{geometry}			% Ajusta o tamanho do papel para A4
\usepackage{titling}					% Permite fazer ajustes no título do artigo
\usepackage{fancyhdr}					% Permite ajustar cabeçalhos e rodapés
\usepackage{indentfirst}				% Ajusta a indentação do primeiro parágrafo
\usepackage{pgfplotstable}
\usepackage{booktabs}
\usepackage{graphicx}
\usepackage[section]{placeins}
\usepackage{float}
\usepackage{listings}
\usepackage{color}

\definecolor{dkgreen}{rgb}{0,0.6,0}
\definecolor{gray}{rgb}{0.5,0.5,0.5}
\definecolor{mauve}{rgb}{0.58,0,0.82}

\lstset{frame=tb,
  language=Python,
  aboveskip=3mm,
  belowskip=3mm,
  showstringspaces=false,
  columns=flexible,
  basicstyle={\small\ttfamily},
  numbers=left,
  numberstyle=\color{gray},
  keywordstyle=\color{blue},
  commentstyle=\color{dkgreen},
  stringstyle=\color{mauve},
  breaklines=fasle,
  breakatwhitespace=true,
  tabsize=3,
  frame=L
}


\graphicspath{{img/}}

%% Define estilo das tabelas %%
\pgfplotstableset {
	every head row/.style={before row=\toprule,after row=\midrule},
    every last row/.style={after row=\bottomrule}
}
            
%% Inicio do documento %%
\usepackage{amsmath}
\begin{document}

%% Título do artigo %%
\begin{titlepage}
\begin{center}

\textsc{\Huge Universidade de São Paulo}\\[5mm]
\textsc{\Large Análise de Desempenho e Eficiência Para Diferentes Variações de Algorítimos de Backtracking}

\vfill
{\Huge SCC0218 \\[2mm] Algoritmos Avançados e Aplicações}
\vfill

\begin{tabular}{rll}
\emph{Professor:}& {\Large Gustavo Enrique de Almeida Prado Alves Batista} \\[5mm]
\emph{Alunos:}& {\Large Ana Carolina Spangler} (8532356) \\[2mm]
& {\Large Paulo B. N. Nascimento} (8531932)
\end{tabular}\\[25mm]

\end{center}
\end{titlepage}

%% Inicio do artigo %%
\section{Análise do Código (github.com/bardes/T1-AA)}
Para implementação do algorítimo o grupo escolheu a linguagem \emph{python},\footnote{Versão 3.4}, com o objetivo de focar na clareza de código e agilidade de desenvolvimento.

Fora criadas duas classes para organizar os dados do programa: \texttt{Node} e \texttt{Graph}. Em conjunto essas classes foram usadas para representar as limitações do domínio do problema através de um grafo representando as fronteiras.

O processamento de fato dos dos dados em busca de uma solução ocorre na função \texttt{SolveNextColor()}, listada abaixo:

\begin{lstlisting}
def SolveNextColor(nodes, policy):
    # Determina o proximo no a ser aprofundado
    target = GetNextNode(nodes, policy)

    # Se nao tem mais nos achou a solucao.
    if target == None:
        return True;

    # Pra cada possivel solucao
    for color in COLORS:
        if nodes.Paint(target, color): # Tenta pintar o no
            if(policy != 'a' and VerifiyPruning(nodes)):
                nodes.Unpaint(target)  # Se podou desfaz a pintura
            elif not SolveNextColor(nodes, policy):#Tenta aprofundar ainda mais
                nodes.Unpaint(target)  # Se nao conseguiu desfaz a pintura
            else:
                return True # Se achou uma solucao retorna True

    # Se nenhuma cor levou a uma solucao, falha.
    return False
\end{lstlisting}

Como é possível notar, a primeira etapa da função é determinar sob qual nó ela deve operar. Esse trabalho é delegado para a função \texttt{GetNextNode()}, que será discutida adiante. Se não foi possível achar nenhum nó para ser trabalhado, isso indica que o problema foi resolvido, e por isso pode-se retornar \texttt{True}, indicando sucesso.

Após determinar o nó a ser trabalhado a função tenta pintá-lo (uma cor de cada vez), sendo que as cores que violariam as limitações do problema são descartadas pela função \texttt{Paint()}, que só permite pintura com cores legais, ou seja, diferente dos vizinhos atuais.

Além disso, dependendo da politica escolhida, pode haver uma etapa adicional de \emph{poda}, que avalia o estado de todos os nós em busca de combinações insolúveis. Caso a verificação de poda detecte algum problema a pintura é desfeita e a próxima cor (se existir) é tentada.

Em seguida o algorítimo tenta aprofundar a solução (recursivamente) chamando \texttt{SolveNextColor()} novamente. Se a tentativa falhara, ou seja, retornar \texttt{False} a escolha de cor atual é desfeita e passa-se para a próxima.

A função \texttt{GetNextNode} é responsável por determinar qual o próximo nó a ser trabalhado, dado o estado atual do mapa e a política escolhida, ou retornar \texttt{None} caso não haja mais nós a se trabalhar.

\begin{lstlisting}
def GetNextNode(nodes, policy):
    if(policy == 'c'):
        opts = sorted(nodes, key = lambda n: \
            (n.Color != None,                   # Em Branco
            len(n.AvailableColors)))            # MRV
        return opts[0].Name if opts[0].Color == None else None

    elif(policy == 'd'):
        opts = sorted(nodes, key = lambda n: \
            (n.Color != None,                   # Em Branco
            len(n.AvailableColors),             # MRV
            -len(nodes.GetNeighbors(n.Name))))  # Desempata com o grau
        return opts[0].Name if opts[0].Color == None else None

    else:
        for node in nodes:
            if node.Color == None:
                return node.Name
        return None
\end{lstlisting}

Como pode-se notar, há 3 possíveis linhas de execução. Caso a política seja \texttt{'a'} ou \texttt{'b'}, a função simplesmente retorna o primeiro nó em branco achado. Já no caso \texttt{'c'} os nós são ordenados colocando-se primeiro os nós em branco, e usando como critério de desempate àquele com menor número de cores possíveis, ou seja, os nós mais restritos primeiro (MRV). Por fim o caso \texttt{'d'} é análogo ao \texttt{'c'}, porém adicionado de um terceiro critério de desempate, colocando-se os nós de maior\footnote{Note o sinal negativo no grau, já que queremos os maiores graus primeiro e a ordenação é em ordem crescente.} grau antes.

\section{Resultados}

Como pode-se notar nos dados das tabela abaixo, as políticas ``A'' e ``B'' variam muito em relação ao desempenho tanto do ponto de vista de total de atribuições, quanto em relação ao tempo de execução. Esse comportamento é esperado, já que cada execução depende da ordem em que os nós são aprofundados, e devido à maneira como conjuntos são implementados em \emph{python}, cada execução apresenta uma ordem diferente.

Mesmo assim é possível estimar que, pelo menos no caso do Brasil, a introdução de \emph{podas} no política ``B'' introduziu ganhos de aproximadamente 3 vezes, tanto na quantidade de atribuições, quanto no tempo de execução, logo os \emph{overheads} compensam os ganhos.

Já a introdução de MVR na política ``C'' trouxe excelentes ganhos, de 3 até 5 ordens de magnitudes em ambas métricas, provando que a heurística MVR sem dúvidas compensa seus \emph{overheads} trazendo ganhos impressionantes.

Finalmente a introdução de um critério de desempate na política ``D'' não pareceu causar benefícios em relação à política anterior, na verdade devido ao \emph{overhead} por causa da adição de um critério extra de desempate essa política parece levar a uma
pequena perda de desempenho em relação à politica ``C''. 

\begin{table}[h!]\centering
	\pgfplotstabletypeset[
		col sep=tab,
		sci zerofill,
		columns/Política/.style={string type}]{bra}
	\caption{Brasil (DP = Desvio Padrão)}
\end{table}

\begin{table}[h!]\centering
	\pgfplotstabletypeset[
		col sep=tab,
		sci zerofill,
		columns/Política/.style={string type}]{eur}
	\caption{Europa (DP = Desvio Padrão)}
\end{table}

\begin{table}[h!]\centering
	\pgfplotstabletypeset[
		col sep=tab,
		sci zerofill,
		columns/Política/.style={string type}]{usa}
	\caption{USA (DP = Desvio Padrão). Política A foi interrompida após 2 horas (7200 segundos) sem resultados. }
\end{table}


\end{document}